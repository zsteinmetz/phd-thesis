% !TeX spellcheck = en_US

\chapter{Summary}

The use of agricultural plastic covers has become common practice for its agronomic benefits such as improving yields and crop quality, managing harvest times better, and increasing pesticide and water use efficiency. However, plastic covers are suspected of partially breaking down into smaller debris and thereby contributing to soil pollution with microplastics.
A better understanding of the sources and fate of plastic debris in terrestrial systems has so far been hindered by the lack of adequate analytical techniques for the mass-based and polymer-selective quantification of plastic debris in soil. The aim of this dissertation was thus to assess, develop, and validate thermoanalytical methods for the mass-based quantification of relevant polymers in and around agricultural fields previously covered with fleeces, perforated foils, and plastic mulches.
\Ac{tga-ms} enabled direct plastic analyses of \SI{50}{\milli\gram} of soil without any sample preparation. With \ac{pet} as a preliminary model, the method \ac{lod} was \SI{0.7}{\gram\per\kilo\gram}. But the missing chromatographic separation complicated the quantification of polymer mixtures. Therefore, \iac{py-gc-ms} method was developed that additionally exploited the selective solubility of polymers in specific solvents prior to analysis. By dissolving \ac{pe}, \ac{pp}, and \ac{ps} in a mixture of \acl{tcb} and \textit{p}-xylene after density separation, up to \SI{50}{\gram} soil became amenable to routine plastic analysis. Method \acp{lod} were \SIrange{0.3}{2.2}{\milli\gram\per\kilo\gram}, and the recovery of \SI{20}{\milli\gram\per\kilo\gram} \ac{pe}, \ac{pp}, and \ac{ps} from a reference loamy sand was \SIrange{86}{105}{\percent}. In the reference silty clay, however, poor \ac{ps} recoveries, potentially induced by the additional separation step, suggested a semi-quantitative evaluation of \ac{ps}.
Yet, the new solvent-based \ac{py-gc-ms} method enabled a first exploratory screening of plastic-covered soil. It revealed \ac{pe}, \ac{pp}, and \ac{ps} contents above \ac{lod} in six fields (\SI{6}{\percent} of all samples). In three fields, \ac{pe} levels of \SIrange{3}{35}{\milli\gram\per\kilo\gram} were associated with the use of \SI{40}{\micro\meter} thin perforated foils. By contrast, \SI{50}{\micro\meter} \ac{pe} films were not shown to induce plastic levels above \ac{lod}. \Ac{pp} and \ac{ps} contents of \SIrange{5}{19}{\milli\gram\per\kilo\gram} were restricted to single observations on four sites and potentially originated from littering. 
The results suggest that the short-term use of thicker and more durable plastic covers should be preferred to limit plastic emissions and accumulation in soil.
By providing mass-based information on the distribution of the three most common plastics in agricultural soil, this work may facilitate comparisons with modeling and effect data and thus contribute to a better risk assessment and regulation of plastics. However, the fate of plastic debris in the terrestrial environment remains incompletely understood and needs to be scrutinized in future, more systematic research. This should include the study of aging processes, the interaction of plastics with other organic and inorganic compounds, and the environmental impact of biodegradable plastics and nanoplastics.