% !TeX spellcheck = en_US

\chapter{Summary}

The use of agricultural plastic covers has become common practice for its agronomic benefits such as improving yields and crop quality, managing harvest times, and increasing pesticide and water use efficiency. However, plastic covers are suspected of partially breaking down into smaller debris and thereby contributing to soil pollution with microplastics.
A better understanding of the sources and fate of plastic debris in terrestrial systems has so far been challenged by the lack of advanced analytical techniques for the mass-based and polymer-selective quantification of plastic debris in soil. The aim of this dissertation thesis was thus to assess, develop, and validate thermoanalytical methods for the exemplary quantification of \ac{pe}, \ac{pp}, and \ac{ps} plastic debris in and around agricultural fields covered with fleeces, perforated foils, and plastic mulches for less than two~years.
Using \ac{tga-ms} enabled direct plastic analyses without any sample preparation. However, the missing chromatographic separation complicated the quantification of polymer mixtures. Therefore, \iac{py-gc-ms} method was developed that additionally exploited the selective solubility of polymers in specific solvents prior to analysis. By dissolving \ac{pe}, \ac{pp}, and \ac{ps} in a mixture of \ac{tcb} and \textit{p}-xylene after density separation, up to \SI{50}{\gram} of soil became amenable to routine plastic analysis. Methodical \acp{lod} were \SIrange{0.3}{2.2}{\micro\gram\per\gram}, and the recovery of \SI{20}{\micro\gram\per\gram} \ac{pe}, \ac{pp}, and \ac{ps} from a reference loamy sand were \SIrange{86}{105}{\percent}.
A first, exploratory screening of plastic-covered soil revealed \ac{pe}, \ac{pp}, and \ac{ps} contents above the \ac{lod} in seven fields (\SI{17}{\percent} of all samples). In three fields, \ac{pe} levels of \SIrange{3}{35}{\micro\gram\per\gram} were associated with the use of thinner and less durable perforated foils (\SI{40}{\micro\meter} thickness). By contrast, \SI{50}{\micro\meter} thick \ac{pe} films were not shown to emit any plastic debris. \Ac{pp} and \ac{ps} contents of \SIrange{5}{19}{\micro\gram\per\gram} were restricted to single observations on four sites and probably originated from littering. 
The results suggest that the short-term use of thicker and more durable plastic covers should be preferred to limit plastic emissions and accumulation in soil.
By providing mass-based information on the distribution of the three main types of plastic debris in agricultural soil, this work may facilitate future comparisons with modeling and effect data, and thus contribute to a better risk assessment and regulation of plastic debris in soil. Yet, the distribution processes and material fluxes of plastic debris in the terrestrial environment remain incompletely understood and need to be scrutinized in future, more systematic research. This particularly applies to aging processes, the interaction of plastics with other organic and inorganic compounds, and the fate and effects of biodegradable plastics and nanoplastics.