% !TeX spellcheck = en_US

\chapter{Final Discussion and Outlook}
\label{ch:general-discussion}

\section{Thermoanalytical Approaches for the Quantification of Plastics in Soil}
\label{sec:general-discussion:analytics}

Mass-based information on the level of plastic pollution in soil is important to link exposure data with modeling and effect data. Up to now, such data has remained largely missing for the lack of appropriate analytical methods. This has emphasized the need for further advances in analytical techniques for the quantification of plastic debris, particularly in heterogeneous samples like soil (Chapters~\ref{ch:plastic-mulching} and \ref{ch:analytical-techniques}).
A first proof-of-principle study using \ac{pet} as a model (Chapter~\ref{ch:tga-ms-method}) demonstrated the potential of thermoanalytics for plastic quantification in soil. The taken \ac{tga-ms} approach permitted the direct analysis of ground soil. This reduced sample preparation to a minimum and thus considerably sped up analysis times with regard to microspectroscopic methods. Microspectroscopic methods normally require an extensive sample preparation including multiple separation, cleanup, and filtration steps (\citealp[for instance,][]{LoderEnzymatic2017}; Chapter~\ref{ch:analytical-techniques}). But the \ac{tga-ms} also involved certain limitations:
\begin{enumerate*}
	\item Sample amounts of about \SI{50}{\milli\gram} hindered the preparation of a representative sample prior to analysis,
	\item the \ac{ms} only detected \ac{pet} contents \SI{>0.7}{\gram\per\kilo\gram}, and
	\item the absence of chromatographic separation limited the simultaneous analysis of different polymers or polymer mixtures with overlapping degradation temperatures (Table~\ref{tab:polymer-decomposition})
\end{enumerate*}.
In the further course of the project, the participation in a round robin test revealed that the developed \ac{tga-ms} method was particularly suitable for distinguishing aromatic polymers like \ac{ps} and \ac{pet} from aliphatic ones such as \ac{pe} and \ac{pp} \citep{BeckerQuantification2020}. The sum of aliphatic and aromatic polymers was each quantified accurately. Such a rough classification may suffice for simple screenings that aim at the identification of distinct plastic hotspots.
Chemometric analyses of \ac{tga-ms} data, for instance, using \ac{pca} and principal component regression may help to pursue a rapid identification and quantification of polymer mixtures by linking \ac{tga} and \ac{ms} data better or by deconvolving interfering signals from other polymers and soil matrix \citep{DavidIntroducing2019}. But this certainly requires further research, which is in line with \citet{MansaThermogravimetric2021} who reasoned that \ac{tga}-based methods are still underutilized for the complexity of the generated data but may provide a robust screening tool for future plastic analyses. However, advances in data analysis will most likely help measurement selectivity more than measurement sensitivity since the latter is physically restricted by the used \ac{ms}.

Using \ac{py-gc-ms} instead allowed for the chromatographic separation of polymer analytes after pyrolysis and their selective quantification as recently acknowledged by \citet{Jimenez-SkrzypekCurrent2021}. The key to the method developed in Chapter~\ref{ch:py-gc-ms-method} was the use of \ac{tcb} to selectively dissolve the target analytes \ac{pe}, \ac{pp}, and \ac{ps} prior to quantification. Solvent-based \ac{py-gc-ms} not only facilitated the preparation of dilution series and sample aliquots but also the extraction of the polymer analytes from the soil matrix. Thereby, up to \SI{4}{\gram} of soil sample were analyzed while decreasing method \acp{lod} to \SIrange{1}{86}{\milli\gram\per\kilo\gram}.
At such low plastic contents though, the method became sensitive to interferences from the soil matrix (Chapter~\ref{ch:intro}). The interferences most likely originated from polymeric \ac{som} fractions that were co-dissolved with the polymer analytes. These polymeric \ac{som} fractions particularly interfered with the quantification of \ac{pe} at levels \SI{<50}{\milli\gram\per\kilo\gram}. \Citet{DierkesQuantification2019} observed similar interferences using \iac{ase}-based \ac{py-gc-ms} approach.

Since this problem could not be tackled from a mere instrumental perspective, the outcome of Chapter~\ref{ch:py-gc-ms-method} involved a critical (re-)evaluation of existing sample preparation and analytical techniques for soil samples (Chapter~\ref{ch:analytical-techniques}). The current literature highlighted the importance of removing soil matrix by density separation and/or oxidative \ac{som} digestion prior to \ac{py-gc-ms} analyses to reduce matrix interferences. However, such sample preparation methods usually require a final filtration step and thereby come at the expense of systematically excluding particles not retained by the filter. These lower size limits are typically \SIrange{1}{10}{\micro\meter} and thus result in a complete loss of the nanoplastics fraction.
But the major finding of the literature research was that analytical methods should aim for sample amounts larger than several grams to account for the particulate nature of discrete plastic debris embedded in a highly heterogeneous soil. The review called for sample preparation techniques specifically tailored to soil, for instance, by including preparative measures for dispersing soil aggregates that potentially occlude plastic debris \citep[Chapter~\ref{ch:intro};][]{ZhangDistribution2018}. Numerical simulations by \citet{YuHow2021} recently underpinned this. The authors recommended the sampling of at least five 1\,\texttimes\,\SI{1}{\square\meter} grid squares to enable the representative quantification\sidenote{Accepting \SI{15}{\percent} variation.} of \num{<=2} homogeneously distributed plastic particles \si{\per\kilo\gram} agricultural topsoil (\SIrange{0}{5}{\centi\meter}). This approximates the minimum amount of plastic debris typically found in agricultural soil \citep{BuksGlobal2020}. A non-homogeneous plastic distribution would already require the sampling of more than \SI{750}{\square\meter} \citep{YuHow2021}. As discussed in Chapter~\ref{ch:analytical-techniques}, the removal of such large quantities of fertile agricultural soil for trace plastic analysis may contradict sustainability efforts and economic interests of farmers and land owners. Hence, practicability and the desired measurement sensitivity need to be carefully balanced.

In the light of these findings, I further refined the solvent-based \ac{py-gc-ms} approach presented in Chapter~\ref{ch:py-gc-ms-method}. This mainly involved the combination of the previously developed method with an efficient yet simple sample preparation (Chapter~\ref{ch:screening}). The overall sample throughput of the complete procedure was \num{25}\,samples per week.
In brief, \SI{50}{\gram} of soil were aggregate\-/dispersed with aqueous sodium hexametaphosphate solution and density\-/separated with saturated \ch{NaCl} solution. Moreover, \textit{p}-xylene was added to the extraction mixture to increase polymer solubility (see Table~\ref{tab:solubility-tests} for solubility tests of an ultra high-density \ac{pe}). Both measures helped to further decrease method \acp{lod} to \SIrange{0.3}{2.2}{\milli\gram\per\kilo\gram} and reduce matrix interferences. At \iac{Corg} content of less than \SI{2.5}{\percent}, matrix interferences were exclusively below the \ac{lod}. The recovery of \SI{20}{\milli\gram\per\kilo\gram} \ac{pe}, \ac{pp}, and \ac{ps} from a reference loamy sand was \SIrange{86}{105}{\percent}. However, recoveries dropped below \SI{70}{\percent} in the reference silty clay or at a spiking level of \SI{2}{\milli\gram\per\kilo\gram}. This was particularly pronounced for \ac{ps} which was hardly detectable in the silty clay suggesting a rather semi-quantitative evaluation of \ac{ps}.
While this clearly defined the quantitative limits of the developed method, it also demonstrated its applicability as a rapid screening tool for the selective quantification of \ac{pe} and \ac{pp}. This is remarkable because the majority of published solvent-based \ac{py-gc-ms} methods have hardly left the stage of method development and validation \citep{DierkesQuantification2019,OkoffoIdentification2020}. As outlined in Chapter~\ref{ch:analytical-techniques}, solvent-based approaches are a constantly evolving field that not only couples batch extractions with \ac{py-gc-ms} (Chapters~\ref{ch:py-gc-ms-method} and \ref{ch:screening}) but keeps exploring combinations of \ac{ase} and \ac{mwe} with \ac{py-gc-ms} or \ac{hnmr} \citep{OkoffoIdentification2020,HermabessiereMicrowaveAssisted2021,NelsonQuantification2019,PeezQuantitative2020}.
Yet, all these mass-based methods remain limited in their application range in terms of the analyzable target polymers and matrices. In this respect, future method refinements will need to increase method robustness and widen the analytical window, for instance, towards biodegradable plastics, particularly clayey soil, and organic-rich environmental matrices like forest soil and compost. This should involve the testing of organic solvents or solvent mixtures for the dissolution of polymers other than \ac{pe}, \ac{pp}, and \ac{ps} as well as adapting and optimizing sample preparation techniques for challenging matrices. In addition, sample preparation techniques for nanoplastics in soil are still to be developed. As for \ac{tga-ms}, emerging machine learning based peak identification and deconvolution algorithms could further support the analysis of \ac{py-gc-ms} data with extraordinarily high interferences or background noise \citep{CowgerCritical2020,MatsuiIdentification2020}.
The typically more time consuming microspectroscopic methods could be applied complementarily, particularly to previously identified plastic hotspots, and provide additional information on particle shapes and sizes (Chapter~\ref{ch:analytical-techniques}).

\section{Plastic Debris in Plastic-Covered Soil}
\label{sec:general-discussion:screening}

The scientific literature reviewed in Chapter~\ref{ch:plastic-mulching} suggested a successive enrichment of plastic debris originating from agricultural plastic covers. However, empirical evidence was scarce when the review was published in 2016 because the required analytical methods were still at an early stage of development. Meanwhile, particle-based plastic screenings have corroborated that initial assumption \citep{HuangAgricultural2020,ZhouMicroplastics2020}. But mass-based information on the level of plastic pollution in agricultural soil remained missing.

Complemented by qualitative \ac{ftir} analyses of debris \SI{>2}{\milli\meter} \citep{CowgerMicroplastic2021}, my refined \ac{py-gc-ms} method enabled a first mass-based assessment of soil-associated \ac{pe} and \ac{pp} plastic debris \SI{<=2}{\milli\meter} in topsoil previously covered with plastic films (Chapter~\ref{ch:screening}). As stated above, \ac{ps} was only evaluated semi-quantitatively for its low recovery.
The screening revealed that the seasonal application of \SI{40}{\micro\meter} thin, perforated films may already lead to a 30-fold increase in soil plastic contents (up to \SI{35}{\milli\gram\per\kilo\gram}) compared to soil covered with thicker films (\SI{50}{\micro\meter}) and fleeces which hardly emitted any plastics above \ac{lod}. \Ac{dsc} and \ac{ftir} analyses of the applied plastic covers further showed first signs of polymer aging that may have triggered the formation of plastic debris.

In general, the plastic levels identified in Chapter~\ref{ch:screening} were about \num{30} times higher than those previously estimated for plastic-covered agricultural soil based on particle counts \citep{BuksGlobal2020} but several orders of magnitude lower than those measured in an exemplary roadside soil \citep{DierkesQuantification2019}. Projections by \citet{BrandesIdentifying2021} indicated that plastic debris emitted from agricultural plastic covers may increase the plastic contents in agricultural soil by \SIrange{5}{9}{\milli\gram\per\kilo\gram} per year.
Although this mostly corresponds to the plastic levels \SI{<=35}{\milli\gram\per\kilo\gram} I found, the equivocal distribution of plastic debris in and around the agricultural fields investigated in Chapter~\ref{ch:screening} did not allow for a reliable identification of plastic sources. This is in line with \citet{CorradiniMicroplastics2021} who screened various land use types for plastic debris at a regional level using \ac{ftir} microspectroscopy. While the authors found the highest numbers of plastic particles on cropland and grassland, they were not able to track down the source of plastic pollution. This is probably due to the ubiquitous use of various kinds of plastic products and their as of yet incompletely understood transport. In addition, most of the studies, including the one presented in Chapter~\ref{ch:screening}, have so far restricted their samplings to the uppermost \SI{5}{\centi\meter} of agricultural or industrial soils, quantified only selected polymers, and excluded nanoplastics. An unequivocal source identification will most likely require a more systematic, in-depth monitoring. This should involve a more representative sampling scheme with a higher number of replicates per transect and larger sample amounts as well as further refined and more robust quantitative methods.

Yet, the insights gained in Chapter~\ref{ch:screening} suggest that the short-term use of thicker and more durable plastic covers should be preferred over thin or perforated foils to limit future plastic emissions and accumulation in soil. For their higher durability, thicker covers have a greater chance of being removed intact after their use, particularly if the soil is moist or has been wetted before retrieval. This, however, requires proper operator training. In this regard, current EU regulations \citep{EN13655Plastics2018} and recycling efforts for agricultural plastics should be intensified to further reduce the amount of plastics inadvertently released into the environment.
Therefore, future research will not only need to systematically trace the major sources of plastic inputs back to different land use systems but also aim for a better understanding of the fluxes of plastic debris in and on soil as well as on landscape level.

\section{Fate of Plastic Debris in Soil and Beyond}
\label{sec:general-discussion:fate}

Soil and plastic debris share common characteristics like their particulate and polymeric structure (Chapter~\ref{ch:intro}). These structural similarities feed back to their fate.
Low-density \ac{som}, for instance, is well-known to preferentially erode from agricultural fields \citep{LalSoil2005,RumpelPreferential2006}. In the same way, fragments of low-density \ac{pe} or \ac{pp} covers ($\rho$ = \SIrange{0.9}{1.0}{\gram\per\cubic\centi\meter}) are transported by wind \citep{RezaeiWind2019,BullardPreferential2021} and water \citep{LaermannsTracing2021,RehmSoil2021} and are thus capable of traveling distances comparable to other non-volatile and water-soluble contaminants \citep{StubbinsPlastics2021}. \Citet{ZhangDistribution2020} estimated that \SI{96}{\percent} of low-density \ac{pe} debris resting on the soil surface (\SI{5}{\degree} hillslope) readily runs off during heavy rainfall events (\SI{>300}{\milli\meter}) while the remaining \SI{4}{\percent} is retained in the soil. By contrast, polymers with higher densities ($\rho$ = \SIrange{1.3}{1.4}{\gram\per\cubic\centi\meter}) are retained better due to gravitational sedimentation \citep{DongTransport2021,O'ConnorMicroplastics2019}.

Once deposited, the plastic debris is likely to age and fragment further (Chapter~\ref{ch:screening}), partially mix with soil, and become entrapped in soil pores or incorporated into soil aggregates \citep{RilligMicroplastic2017}. \Citet{ZhangDistribution2018} found \SI{72}{\percent} of plastic debris (\SIrange{0.05}{1}{\milli\meter}) associated with soil aggregates of an arable soil. The polymeric nature of both plastics and \ac{som} probably facilitates their mutual intermolecular stabilization and heteroaggregation \citep{SchaumannSoil2006,LuoDistribution2020}. Dependent on the polarity of the respective polymer, electrostatic or van der Waals interactions may prevail \citep{LuoDistribution2020}. Due to an increased surface-to-volume ratio, small particles are most likely better stabilized than larger particles. In line with this, field-scale rainfall simulations showed that \SIrange{250}{300}{\micro\meter} high-density \ac{pe} particles eroded faster than those of \SIrange{50}{100}{\micro\meter} size \citep{RehmSoil2021}. Column experiments by \citet{WuTransport2020} indicated that \ac{ps} nanoplastics (\SI{100}{\nano\meter} spheres) are effectively retained by \ch{Ca^{2+}} and \ch{Fe} or \ch{Al} oxides but may be remobilized at a pH \num{>9}.

Advective transport of plastic particles through the soil column was not observed in Chapter~\ref{ch:screening} since the soil sampling was restricted to the uppermost \SI{5}{\centi\meter}. However, vertical plastic transport is often \SI{<50}{\percent} lower than conservative tracers and thus mostly limited to the uppermost \SIrange{10}{25}{\centi\meter} of the soil anyway \citep{KellerTransport2020,WuTransport2020}. Instead, plastic debris is rather assumed to be transported to deeper soil via tillage, bioturbation, and preferential flow through soil cracks and earthworm burrows \citep{RilligMicroplastic2017a,YuLeaching2019,LiVertical2021}. Preferential flow may eventually make plastic particles reach groundwater where saturated conditions could promote their mobility\sidenote{See \citet{RenMicroplastics2021} for a review on plastic fate at the soil--groundwater interface.}. Under such saturated conditions, spherical particles were shown to be more mobile than fragmented or fibrous ones, particularly if the size of the plastic particles is lower than that of the medium \citep{WaldschlagerInfiltration2020}. Worst-case projections by \citet{WaldschlagerInfiltration2020} indicated a maximum infiltration depth of spheres in medium gravel of about \SI{2}{\meter} when applying an immense water flow of \SI{150}{\liter\per\minute\per\square\meter} for \SI{1}{\hour}. Plastic fragments and fibers infiltrated less than \SI{10}{\centi\meter} into matrices with smaller particles sizes like fine gravel, medium sand, or coarse silt.
This may also explain why elevated microplastic levels of up to \num{80}\,particles\,\si{\per\liter} groundwater have so far only been detected in the immediate vicinity of heavily contaminated sites such as landfills \citep{ManikandaBharathSpatial2021}. Groundwater wells were virtually free of plastics \citep{MintenigLow2019}.

Hence, current research suggests that:

\begin{itemize}
	\item Plastic particles larger than soil particles and less dense than the soil bulk density are preferentially eroded.
	\item Smaller plastic debris with a density close to the bulk density of the soil may be retained more easily in the soil.
	\item Plastic retention in soil limits further transport and favors accumulation.
	\item Nanoplastic mobility depends on the prevailing soil physicochemical properties such as ionic strength, pH, and the presence of \ch{Fe} or \ch{Al} oxides.
\end{itemize}

This indicates that the low-density plastics \SI{>2}{\milli\meter} identified in Chapter~\ref{ch:screening} were probably rather mobile when released from the plastic covers. Upon fragmentation into debris \SI{<=2}{\milli\meter}, they may become increasingly incorporated into the bulk soil and accumulate there.
The continuous use of agricultural plastic covers as well as sewage sludge applications were in fact already shown to accumulate plastic debris in agricultural topsoil \citep{HuangAgricultural2020,CorradiniEvidence2019}\sidenote{See \citet{BuksGlobal2020} for a comprehensive review on plastic debris in soil.}. In this sense, plastic debris may become an artificial \ac{som} fraction in the long term. \Citet{StubbinsPlastics2021}, for example, recently numbered plastic debris among soil \ac{Corg} stocks. This concerns aged plastics in particular that will become more similar to the surrounding soil with time (Table~\ref{tab:polymer-aging}).
On the contrary, \citet{RilligMicroplastic2018} argued that plastic debris is too different in its physicochemical properties and function to be considered a part of \ac{som}. Due to the recalcitrance of conventional plastics towards degradation, they will, if at all, only participate in the soil's long-term carbon cycle, probably comparable to the pyrogenic biochar components of \latin{Terra preta} \citep{RilligMicroplastic2021}. The contemporary contribution of plastic debris to soil functions and quality may, however, be limited if not detrimental. Moreover, changing environmental conditions and agricultural practices like tillage have the potential of remobilizing deposited plastics. This may particularly apply to nanoplastics which are still largely understudied.
Apart from the plastics themselves, their capability of releasing additives and other polymer-associated compounds into the environment potentially contributes to the adverse effects to soil quality.

\begin{margintable}[-14\baselineskip]
	\centering\footnotesize
	\caption[Changes in polymer properties while aging.]{Changes in polymer properties while aging \citep{RenMicroplastics2021}.}\label{tab:polymer-aging}
	\begin{tabular}{lc}
		\toprule
		Surface roughness & + \\
		Microcracks & + \\
		Tensile strength & \textminus \\
		Crystallinity & + \\
		Polarity & + \\
		Molecular weight & \textminus \\
		Functional groups & + \\
		(\ch{COOH}, \ch{C=O}, \ch{C-OH}, \ch{=CH}) \\
		Leaching of additives & + \\
		Sorption capacity  & + \\
		\bottomrule
	\end{tabular}
\end{margintable}

\section{Effects of Plastic Debris on Soil Quality}
\label{sec:general-discussion:effects}

While the effects of agricultural plastic covers on soil quality are extensively discussed in Chapter~\ref{ch:plastic-mulching}, the potential effects of plastic debris accumulating in soil were virtually unknown at the time of writing.
Current reviews point out that plastic debris has the ability of changing the soil's microstructural environment which may lead to adverse effects on the soil microbial community, plant growth, net primary production, and litter decomposition \citep{RilligMicroplastic2021,MbachuRise2021,QiBehavior2020}. \Citet{ZhangSystematical2021,RilligMicroplastic2021} reasoned that such effects may be either direct or indirect, namely mediated by changes in the soil's physicochemical properties. \Citet{deSouzaMachadoMicroplastics2019}, for instance, spiked a loamy sand at \SIlist{2;20}{\gram\per\kilo\gram} \ac{pet} fibers, \ac{pa} beads, and \ac{pe}, \ac{pp}, \ac{ps}, and \ac{pet} fragments (\SIrange{8}{5000}{\micro\meter} size) and assessed various soil quality criteria after \SI{12}{weeks} of incubation. Fragments and fibers decreased the soil bulk density by up to \SI{20}{\percent} and thereby significantly altered the soil structure. The water holding capacity increased by \SIrange{10}{40}{\percent}; the effect was most pronounced for \ac{pet} fibers. Evapotranspiration and the fraction of water stable aggregates were reduced by up to \SIlist{60;25}{\percent}, respectively, in the presence of \ac{pet} fibers and \ac{pa} beads. These changes in the soil water dynamics are probably \ac{som}-dependent \citep{ZhangVariations2020,LiangEffects2021} and propagated to an increase in soil microbial activity\sidenote{See also \citet{deSouzaMachadoImpacts2018}.} and root and bulb growth of \latin{Allium fistulosum}, the spring onion \citep{deSouzaMachadoMicroplastics2019}. In general, the effects observed by \citet{deSouzaMachadoMicroplastics2019} were more pronounced for fibers and beads whose shape considerably differs from natural soil particles\sidenote{See also \citet{RilligMicroplastic2019,LehmannMicroplastics2021}.}. Film debris was not assessed. But a meta analysis by \citet{GaoEffects2019} found adverse effects of residual plastic films on crop yield at levels \SI{>24}{\gram\per\square\meter} agricultural soil. Although inconsistent units impede detailed comparisons, this may approximate \SIrange{0.4}{1.0}{\gram\per\kilo\gram} when assuming a homogeneous plastic distribution in the uppermost \SI{5}{\centi\meter} soil layer with a bulk density range of \SIrange{0.9}{2.0}{\gram\per\cubic\centi\meter} \citep{HornPhysical2016}.
The effect levels reported by \citet{deSouzaMachadoMicroplastics2019,GaoEffects2019} are thus about \numrange{1}{3} orders of magnitude higher than the plastic contents I measured in previously plastic-covered soil (\SI{<=35}{\milli\gram\per\kilo\gram}, Chapter~\ref{ch:screening}). This indicates a limited impact of the found plastic debris on soil quality and productivity. However, it is worth noticing that my solvent-based \ac{py-gc-ms} approach only covered \ac{pe}, \ac{pp}, and \ac{ps}, which are the most abundant polymers (Chapter~\ref{ch:intro}), but may have underestimated total polymer contents.

\Citet{BuksWhat2020} comprehensively reviewed the current state of research on the effects of plastic debris towards the multicellular soil fauna. The authors inferred that nematodes, gastropods, and rotifers responded the most sensitively to plastic levels \SI{<=100}{\milli\gram\per\kilo\gram} \citep[for instance,][]{KimSizedependent2020,SongUptake2019}. By contrast, collembolans and earthworms were more robust and required barely realistic plastic contents of \SI{>1}{\gram\per\kilo\gram} to induce adverse effects \citep{JuEffects2019,DingEffect2021,LahiveMicroplastic2019}. Beetles, termites, ants, and mites remained largely unaffected by plastic debris \citep[for instance,][]{PengBiodegradation2019,ZhuTrophic2018}. The observed effects were mostly sublethal and included changes in the gut microbiome, oxidative stress, metabolic malfunctioning \citep{JuEffects2019,ChenDefense2020,Rodriguez-SeijoOxidative2018}, avoidance \citep{DingEffect2021}, or a reduced motility, growth, or reproduction \citep{BootsEffects2019,LahiveMicroplastic2019}. In line with the findings by \citet{deSouzaMachadoMicroplastics2019} discussed above, smaller and more irregularly shaped particles like fragments and fibers tended to be more ecotoxicologically active than larger and spherical particles; yet, \ac{ps} microbeads still dominate effect studies \citep{BuksWhat2020}. The polymer type appears to be less influential \citep{RilligMicroplastic2020}, which may be different for nanosized particles, though \citep{RilligMicroplastic2019}. However, particle shapes and sizes were not assessed in Chapter~\ref{ch:screening} and would have required a complementary particle-based analysis.

\citet{BahoMicroplastics2021} criticized that the vast majority of effect studies have focused on single species so far and were rather short-term (\SI{30}{\day}). The authors advocated for longer-term multispecies and ecosystem level studies to prevent an ``ecological surprise''; this is when ecosystems behave fundamentally differently than previously anticipated from smaller-scale laboratory or mesocosm studies. Despite the desire for more realistic testing, the majority of studies applied plastic contents far beyond realistic exposure levels \citep{BuksWhat2020}. The actual contribution of plastic debris to the degradation of soil ecosystem and functioning therefore remains incompletely understood. This uncertainty is also because effect studies typically refer to plastic contents on a mass basis while the greater part of exposure studies reports particle counts \citep{LeuschConverting2021}. Here, the mass-based \ac{py-gc-ms} method developed in Chapters~\ref{ch:py-gc-ms-method} and \ref{ch:screening} might be particularly useful for the harmonization of effect and exposure data.

A first risk assessment recently conducted by \citet{JacquesProbabilistic2021} estimated environmental exposure distributions ($n = 48$) and species sensitivity distributions ($n = 37$) by converting published no and lowest observed effect particle masses to counts. Contrary to the reasoning by \citet{BuksWhat2020}, particle sizes or shapes were not considered. The hazardous plastic levels at which \SI{5}{\percent} of species would be affected were \numlist{230;160}\,particles\,\si{\per\kilo\gram} based on no and lowest observed effect levels, respectively. These species were \latin{Caenorhabditis elegans}, a model nematode, garden cress (\latin{Lepidium sativum}), and \latin{Aspergillus flavus}, a pathogenic fungus. The hazardous levels were exceeded in about \SI{60}{\percent} of the investigated exposure studies. At \num{8100}\,particles\,\si{\per\kilo\gram}, this is the \num{95}th percentile of the environmental exposure distribution, \SIrange{22}{28}{\percent} of species would be affected. The three sites exceeding that level were industrial, urban, and agricultural.
While this is certainly alarming, the results contrast previous mass-based studies that scarcely identified adverse effects on soil organisms at realistic exposure levels \citep{BuksWhat2020}. The distributions calculated by \citet{JacquesProbabilistic2021} highly depend on the quality of the input data and are subject to great uncertainties due to the underlying mass-to-particle conversions as emphasized in Chapters~\ref{ch:analytical-techniques} and \ref{ch:screening}. For example, some of the included studies conducted their ecotoxicity tests in aqueous soil solution, did not state the polymer used, or reported plastic levels per unit area.
Therefore, more systematic and unit-harmonized research is indispensable for a comprehensive eco(toxico)logical risk assessment. This should then also acknowledge plastic shapes, sizes, and aging and include interactive effects with other stressors like polymer-associated compounds.

\section[Interaction of Plastic Debris with Plastic-Associated Compounds]{Interaction of Plastic Debris with Additives, Agrochemicals, and other Plastic-Associated Compounds}
\label{sec:general-discussion:pacs}

The interaction of plastic debris with organic substances, be it additives or (agro)\-chemicals from external sources, is still understudied and to some extent contradictory\sidenote{See \citet{XiangMicroplastics2022} for a detailed overview.}.
In fact, the majority of polymers including agricultural plastic covers contain additives like color pigments, antioxidants and light stabilizers, plasticizers, lubricants, or flame retardants (Chapters~\ref{ch:plastic-mulching}). But contrary to the assumptions made in Chapter~\ref{ch:plastic-mulching}, the qualitative additive screening of the plastic covers via \ac{ted-gc-ms} (Chapter~\ref{ch:screening}) did not show any \acp{pae} but \ac{uv} stabilizers and lubricants only. This is not surprising as \acp{pae} are plasticizers mostly added to \ac{pvc} rather than \ac{pe} or \ac{pp} \citep{WaltersPlasticizers2020}. Lubricants may facilitate spreading the covers out on site and \ac{uv} stabilizers prolong their lifetime when exposed to sunlight (Chapters~\ref{ch:plastic-mulching} and \ref{ch:screening}).
However, leaching of additives from the polymer backbone and their release into the surrounding soil are often hypothesized \citep[Chapter~\ref{ch:plastic-mulching};][]{PathanSoil2020,ZhangTransport2020} and modeled \citep{ZhangAgricultural2021} yet hardly observed in field studies \citep{QiBehavior2020}.
\Citet{LiAre2021}, for instance, found up to \num{5} times higher \ac{pae} contents in greenhouse soil than in non-covered soil. But no clear correlation was established with the amount or type of plastic debris found, suggesting another input source or factor driving their distribution\sidenote{See also \citet{BillingsPlasticisers2021} for a comprehensive review on plasticizers in the terrestrial environment.}. This could be seed coatings, impure agrochemicals, or changing environmental conditions remobilizing legacy contaminations.
Other additives than \acp{pae} are typically added to polymers in much smaller quantities \citep{HahladakisOverview2018} so that their detection after desorbing and dispersing into the surrounding is even more challenging (Chapter~\ref{ch:screening}).

In contrast to polymer additives desorbing from the polymer backbone, sorption of organic compounds to plastic particles has been more systematically investigated. \Citet{HufferSorption2016}, for example, studied the sorption behavior of numerous apolar, monopolar, and bipolar organic pollutants to \ac{pe}, \ac{pvc}, \ac{pa}, and \ac{ps} particles in aqueous solution. Aromatic compounds were shown to preferentially interact with \ac{ps} via $\pi$--$\pi$ stacking. Interactions with \ac{pe} were mostly driven by migration of the organic compounds into the polymer bulk phase, while surface interactions dominated with \ac{pa} and \ac{pvc}. Similarly, \iac{pe} mulch sorbed about \SI{23}{\percent} of various agrochemicals in aqueous solution which additionally slowed down their degradation by \SI{30}{\percent} \citep{BeriotLaboratory2020}. The sorption potential correlated well with the \ac{kow} of the agrochemicals \citep{BeriotLaboratory2020,WangAdsorption2020,SuntaAdsorption2020,LanComparative2021}. The molecular mechanisms controlling the sorption process were hydrophobic partitioning and electrostatic forces \citep{TourinhoPartitioning2019,LanComparative2021}.
However, \citet{RamosPolyethylene2015} already showed that migration of pesticides to\slash from \ac{pe} covers may be bidirectional. The state of the dynamic equilibrium depends on the organic substance of interest and its polarity (\ac{kow}) as well as the polymer type, the degree of polymer aging, interactions with other environmental compartments, and prevailing environmental conditions. Diffusion models by \citet{CastanMicroplastics2021}, for instance, indicated that the sorption of organic contaminants to plastics is only stable for sorbates with a $\log$ \ac{kow} >5. An increase in the soil water content, however, mobilized the disinfectant triclosan ($\log$ \ac{kow} = 4.8) previously sorbed to \ac{pe} \citep{ChenComparison2021}.

In aqueous solution, \ac{uv}-aged \ac{ps} particles sorbed up to \num{10} times less organic pollutants than virgin particles \citep{HufferData2018,HufferSorption2018}. This was attributed to surface oxidation of the polymer surface as indicated by an increased number of carbonyl moieties. By contrast, the broad-spectrum insecticide fibronil sorbed the best to polymers containing carbonyl groups \citep{GongComparative2019}.

An agricultural sandy loam spiked at \SI{100}{\gram\per\kilo\gram} \ac{pe} reduced the soil's overall sorption capacity towards the herbicides atrazine and 2,4-D by half compared to blank soil \citep{HufferPolyethylene2019}. However, this effect was most pronounced at pH~\numrange{3}{5}, and \ac{pe} contents were particularly high. In line with this, \SI{100}{\gram\per\kilo\gram} \ac{pe}, \ac{pp}, and \ac{ps} decreased the sorption of diazepam, a sewage sludge\-/borne anxiolytic, to sandy loam and silty clay \citep{XuContrasting2021}. At \SI{10}{\gram\per\kilo\gram} \ac{pe}, \ac{pp}, and \ac{ps}, however, the sorption potential of phenanthrene increased. Phenanthrene sorbed the best to \ac{pe}, followed by \ac{som}, \ac{pp}, and \ac{ps}. Polymers pretreated with \ac{som} were decreased in their sorption capacity \citep{XuContrasting2021}.
By contrast, \citet{AteiaSorption2020} assessed the interaction of atrazine, acetamidophenol, and two perfluoroalkyl compounds with eight different polymers and kaolin in the presence of dissolved organic matter. The sorption to kaolin was generally about one order of magnitude lower than the sorption to polymers. Recycled plastics and polymers inoculated for two weeks with organic matter sorbed a higher amount of organic substances than virgin polymers.
In line with this, sorption to \ac{pe} debris reduced the adverse effects of phenanthrene and anthracene towards bacteria \citep{KleinteichMicroplastics2018}. On the contrary, the bioconcentration of perfluoroalkyl compounds in earthworms doubled in the presence of \ac{pvc} particles \citep{SobhaniMicroplastics2021}.

Current literature indicates that the desorption and release of polymer additives into the surrounding soil requires more basic research. The sorption of other, non-additive organic compounds like agrochemicals or pharmaceuticals to plastic particles is understood better but their competitive sorption potential in the presence of different soil constituents and polymer types should be assessed more systematically. All in all, the sorption behavior of organic substances to plastic particles seem to depend mainly on the polarity of the sorbing organic substance (\ac{kow}), the quantity and type of the polymer present, and the soil characteristics including its texture, water content and cation exchange capacity. For more comprehensive quantitative structure--activity relationships, other important physicochemical parameters like the surface tension or the specific surface area of the soil, sorbent polymers, and potential sorbates should be investigated further.
In comparison to the plastic contents reported in Chapter~\ref{ch:screening} (\SI{<=35}{\milli\gram\per\kilo\gram}), sorption studies have so far used \num{>20} times higher plastic contents. As opposed to the assumptions made in Chapter~\ref{ch:plastic-mulching}, such low plastic contents are suggested to have only a limited influence on the soil's overall (de)sorption potential. Nanoplastics with particularly high surface areas may need to be separately assessed though. Moreover, it is still unknown how plastic aging may affect the fate of polymer-associated compounds in the long term.

\section{Towards a Sustainable Use of Agricultural Plastic Covers?}
\label{sec:general-discussion:sustainable}

Given the current uncertainties about the long-term fate and effects of plastic debris in the terrestrial ecosystem, a sustainable agriculture may be advised to act according to the precautionary principle \citep{RhodesPlastic2018,BackhausMicroplastics2020,MollerFinding2020}. This is not only to protect the environment but also agronomic revenues. \Citet{BertlingKunststoffe2021} recently stated that German agricultural fields with plastic levels \SI{>1}{\gram\per\kilo\gram} become unattractive to farmers and thus practically lose their agronomic value. This coincides with the effect levels discussed in Section~\ref{sec:general-discussion:effects}. However, subtle or long-term effects on the soil ecosystem may already occur at plastic levels \SI{<1}{\gram\per\kilo\gram}. Overstepping certain, yet unknown ``microplastic tipping points'' is to be avoided \citep{QiBehavior2020}.

In order to limit future plastic emissions into the environment, \citet{ThompsonPlastics2009,ScalengheResource2018,RhodesPlastic2018} proposed a holistic systems thinking approach that integrates plastic production, consumption, and disposal into a sustainable circular economy. As discussed in Chapter~\ref{ch:plastic-mulching}, this means:

\begin{enumerate}
	\item Reducing plastic products and, if possible, replacing them with viable alternatives.
	\item (Re)using plastic materials as long as possible.
	\item Recycling the plastics at their end of life.
\end{enumerate}

\Citet{SimonBinding2021} additionally advocated for a greater durability of newly produced plastic products that would virtually eliminate single use, an increased product safety, as well as a transparent labeling of biodegradable and bio-based plastics.

Simply reducing the application range of agricultural covers made from conventional plastics would probably decrease the global agricultural production (Chapter~\ref{ch:plastic-mulching}). Therefore, replacing conventional plastics with biodegradable alternatives or other, natural materials seems more practicable \citep{BrandesMikro2020} and is, in fact, increasingly implemented. Agricultural covers made of biodegradable polymers like \ac{pla} or starch blends already capture a market share of \SI{7}{\percent} in Germany \citep{BertlingKunststoffe2021}. However, the main challenge in the development of biodegradable agricultural covers remains: This is making them resistant to degradation while in use but ensuring their complete degradation under realistic environmental conditions at the end of life \citep[Chapter~\ref{ch:plastic-mulching};][]{BertlingKunststoffe2021}. The use of biodegradable plastic covers thus continues to bear the risk of leaving incompletely degraded plastic debris in the soil after the growing season \citep{SanderBiodegradation2019,VieraAre2021}.

At the same time, conventional \ac{pe} mulches covering asparagus or strawberry cultivations (\SI{>50}{\micro\meter} film thickness) are already reused for up to \SI{10}{years}. Thinner perforated foils or fleeces have shorter life times and are often replaced after one growing season \citep[Chapter~\ref{ch:screening};][]{BertlingKunststoffe2021}. Moreover, their low durability makes it particularly likely that film residues stay on the field. Therefore, the use of thicker covers may not only reduce the formation of plastic debris but also their potential for reuse.

The recycling of used plastic covers was challenging in the past for their contamination with soil and agrochemicals (Chapter~\ref{ch:plastic-mulching}). However, advances by the German recycling system ``ERDE'' recently enabled the retrieval, collection, and mechanical recycling of \SI{50}{\percent} of all agricultural plastic covers used across the country \citep{BertlingKunststoffe2021,ERDERecyclingERDE2021}. Ideally, such initiatives should be extended to other countries in the future. China, for instance, still consumes the vast majority of agricultural plastics \citep[Chapter~\ref{ch:plastic-mulching};][]{MormileWorld2017}.

Life cycle assessments show that conventional, petrochemically produced \ac{pe} mulches emit \SIlist{2.7;3.6}{\kilo\gram} \ch{CO2} \si{\per\kilo\gram} mulch when landfilled or incinerated, respectively. Mechanical recycling has a carbon footprint of \SI{1.2}{\kilo\gram} \ch{CO2} \si{\per\kilo\gram} \ac{pe} mulch \citep{BosLife2008}. In contrast, biodegradable and bio-based alternatives like \ac{pla} have the potential to close material cycles and thus reduce their carbon footprint to up to \SI{1.7}{\kilo\gram} \ch{CO2} \si{\per\kilo\gram} \citep{KollerSwitching2019,RezvaniGhomiLife2021}. At least for \ac{pla}, however, such a low carbon footprint is achieved only if the \ac{pla} is mechanically recycled instead of being left for degradation. This is because polymer production is still energy intensive and may cause \ch{CO2} emissions of up to \SI{2.8}{\kilo\gram\per\kilo\gram} \ac{pla} \citep{RezvaniGhomiLife2021,AltmanMyth2021}. Further research and development is thus needed to make the production of biodegradable and bio-based polymers more energy efficient \citep{VieraAre2021}. This may also include the intelligent design of novel polymers that are optimized towards low energy production and recycling \citep{VieraAre2021,KorleyPolymer2021,KakadellisAchieving2021} while avoiding competition with food supplies \citep{RhodesPlastic2018}. Moreover, it is still largely unknown how and to what extent biodegradable agricultural covers and plastic debris at various stages of decomposition may affect the soil ecosystem \citep{SanderBiodegradation2019,QinReview2021,AltmanMyth2021}. Adding fresh and easily available carbon sources to soil is well known to stimulate the decomposition of older and more recalcitrant \ac{som}; a process called ``priming'' \citep{ChenMixing2020}.

For agricultural plastics escaping the material cycle and entering the environment, \citet{BertlingKunststoffe2021} suggested the implementation of some sort of ``emergency degradation'' that ensures the decomposition of plastics lost in the field for at least \SI{50}{years}. This, as well, would imply the development of new polymers or the modification of biodegradable polymers to resist degradation for a longer period of time. On the contrary, \citet{ScalengheResource2018} raised the question whether it may be generally acceptable to leave certain amounts of chemically-inert plastics in the environment. This idea may have been motivated by the fact that the remediation and removal of plastic debris from soil is still at an early stage of development \citep{PadervandRemoval2020} and will most likely come with potential unwanted side effects.

For the lack of ready-to-use alternatives, a sustainable use of agricultural plastic covers should first and foremost aim to reduce plastic emissions by using fewer or more durable plastic covers that prevent the emission of plastic debris and thus long-term plastic accumulation in the environment (Chapter~\ref{ch:plastic-mulching}).
The transition to a more plastic-aware agriculture could be guided by targeted policy and regulatory measures. While the EU has already banned agricultural plastic covers thinner than \SI{20}{\micro\meter} \citep[Chapter~\ref{ch:screening};][]{EN13655Plastics2018}, \citet{StubenrauchPlastic2020} recently elaborated that the current ``command-and-control legislation'' by the EU and Germany insufficiently protects agricultural soil from plastic emissions. The authors came to the conclusion that the current legislation particularly neglects the monitoring and regulation of plastic debris \SI{<2}{\milli\meter} which has so far been justified with restricted analytical capabilities. The analytical advances outlined in this dissertation (Chapters \ref{ch:tga-ms-method}--\ref{ch:screening}) may thus eventually contribute to a better regulation of plastic debris in soil.

\section{Concluding Thoughts}
\label{sec:general-discussion:conclusion}

The aim of this dissertation was to scrutinize the extent to which agricultural plastic covers emit plastic debris into the surrounding soil. To this end, thermoanalytical techniques were assessed, developed, and validated. The final solvent-based \ac{py-gc-ms} method enabled the simultaneous and selective routine analysis of \ac{pe} and \ac{pp} debris in soil. \Ac{ps} was only assessed semi-quantitatively for its low recovery. For \ac{pe} and \ac{pp}, the method had the clear advantage of delivering highly sensitive mass-based information that could be readily compared with modeling or effect data. Applying the method in a first screening study indicated indeed that plastic covers were associated with elevated \ac{pe} levels in agricultural topsoil, particularly if thinner and less durable films were applied. However, the ubiquitous use of plastic products made it difficult to pinpoint plastic covers as \emph{the} source of plastic debris. Moreover, the solvent-based \ac{py-gc-ms} approach systematically excluded other polymers than \ac{pe}, \ac{pp}, and \ac{ps} as well as nanoplastics so that total plastic levels may be higher. Yet, the found plastic levels were in agreement with previous reports but still several orders of magnitude lower than effect levels.
At the current state of research, the risk of plastic debris to soil organisms thus appears to be limited but may increase with more and nanosized plastics accumulating in the environment. This calls for precaution particularly because current literature suggests that adverse effects of plastic debris on soil life and quality largely depend on the size and shape of the plastics. Particle shapes and sizes cannot be assessed with the mass-based approach taken in this thesis and require complementary microspectroscopy of potential hotspots.
Furthermore, the distribution processes and material fluxes of plastic debris in the terrestrial environment are still incompletely understood and need to be assessed in future, more systematic research. This should extend to other polymers including biodegradable plastics and nanoplastics which still lack appropriate analytical techniques.
