% !TeX spellcheck = en_US
% TODO: Verweise auf Kapitel

\chapter{General Discussion and Outlook}
\label{ch:general-discussion}

\section{Review of Scientific Insights}
\label{sec:general-discussion:review}

The scientific literature reviewed in Chapter~\ref{ch:plastic-mulching} suggested a successive enrichment of plastic debris originating from agricultural plastic covers together with a potential release of polymer additives like \acp{pae}. However, empirical data were scarce at that time because the required analytical methods were still at an early stage of development. This emphasized the need for further advances in analytical techniques for the quantification of plastic debris in heterogeneous samples like soil.

The subsequently conducted proof-of-principle study using \ac{pet} as a model (Chapter~\ref{ch:tga-ms-method}) demonstrated the potential of thermoanalytics for plastic quantification in soil. Thermoanalytical methods are mass-based and thus enable direct comparisons with modeling or effect data. In addition, the used \ac{tga-ms} provided the opportunity of reducing sample preparation to a minimum, but it also involved certain limitations:
\begin{enumerate*}
	\item Sample amounts of about \SI{50}{\milli\gram} challenged the preparation of a representative sample prior to analysis,
	\item the \ac{ms} only detected \ac{pet} contents in the \si{\milli\gram\per\gram} range, and
	\item missing chromatographic separation limited the analysis of polymer mixtures
\end{enumerate*}.
In the further course of the project, the participation in a round robin test revealed that the \ac{tga-ms} method was particularly suitable for distinguishing aromatic polymers like \ac{ps} and \ac{pet} from aliphatic ones such as \ac{pe} and \ac{pp}. The sum of aliphatic and aromatic polymers was quantified accurately \citep{BeckerQuantification2020}. The chemometric analysis of \ac{tga-ms} data using \ac{pca} and principal component regression may help to pursue a rapid identification and quantification of polymer mixtures \citep{DavidIntroducing2019} but still requires further research.
% TODO: Add \citep{MansaThermogravimetric2021}

Using \ac{py-gc-ms} instead allowed for the chromatographic separation of polymer analytes after pyrolysis and their selective quantification as recently acknowledged by \citet{Jimenez-SkrzypekCurrent2021}. The key to the method developed in Chapter~\ref{ch:py-gc-ms-method} was the use of \ac{tcb} to selectively dissolve the target analytes \ac{pe}, \ac{pp}, and \ac{ps} prior to quantification. Solvent-based \ac{py-gc-ms} not only facilitated the preparation of dilution series and sample aliquots but also the extraction of the polymer analytes from the soil matrix. Thereby, up to \SI{4}{\gram} of soil sample were analyzed while decreasing \acp{lod} to the \si{\micro\gram\per\gram} range.
At such low plastic contents and opposed to the assumptions made in Chapter~\ref{ch:intro}, the method became sensitive to interferences from the soil matrix. The interferences most likely originated from polymeric \ac{som} fractions (\ac{Corg} \SI{>2.5}{\percent}) that were co-dissolved with the polymer analytes. These polymeric \ac{som} fractions particularly interferred with the quantification of \ac{pe} at levels \SI{<50}{\micro\gram\per\gram}.

This unexpected outcome involved a critical (re-)evaluation of existing sample preparation and analytical techniques for soil samples (Chapter~\ref{ch:analytical-techniques}). The current literature indeed highlighted the importance of removing soil matrix by density separation and/or oxidative \ac{som} digestion to reduce matrix interferences. However, such sample preparation methods usually require a final filtration step and thereby come at the expense of systematically excluding particles not retained by the filter. These lower size limits are typically \SIrange{1}{10}{\micro\meter} and thus involve a complete loss of the nanoplastics fraction.
But the major finding of the literature research was that analytical methods should aim for sample amounts larger than several grams to account for the particulate nature of discrete plastic debris embedded in the highly heterogeneous soil. The review further called for sample preparation techniques specifically tailored to soil samples, for instance, by including preparative measures for dispersing soil aggregates that occlude plastic debris \citep{ZhangDistribution2018}. Numerical simulations by \citet{YuHow2021} recently underpinned this. The authors recommended the sampling of at least five 1 \texttimes\ \SI{1}{\square\meter} grid squares to enable the representative quantification (accepting \SI{15}{\percent} variation) of \num{<=2} homogeneously distributed plastic particles \si{\per\kilo\gram} agricultural topsoil (\SIrange{0}{5}{\centi\meter}). This approximates the minimum amount of plastic debris typically found in agricultural soil \citep{BuksGlobal2020}. A non-homogeneous plastic distribution would already require the sampling of more than \SI{750}{\square\meter} \citep{YuHow2021}. As discussed in Chapter~\ref{ch:analytical-techniques}, the removal of such large quantities of fertile agricultural soil for trace plastic analysis may contradict sustainability efforts and economic interests of farmers and land owners. Hence, practicability and the desired measurement sensitivity need to be carefully balanced.

In the light of these findings, I further refined the solvent-based \ac{py-gc-ms} approach. This mainly involved the combination of the previously developed method with an efficient yet simple sample preparation (Chapter~\ref{ch:screening}). To this end, \SI{50}{\gram} of soil were aggregate\-/dispersed with aqueous sodium hexametaphosphate solution and density\-/separated with saturated \ch{NaCl} solution. Moreover, \textit{p}-xylene was added to the extraction mixture to increase polymer solubility. Both measures helped to decrease \acp{lod} to \SIrange{0.3}{2.2}{\micro\gram\per\gram} and reduce matrix interferences. At \iac{Corg} content of less than \SI{2.5}{\percent}, matrix interferences were exclusively below the \ac{lod}.
Complemented by qualitative \ac{ftir} analyses of debris \SI{>2}{\milli\meter} \citep{CowgerMicroplastic2021}, the refined method allowed for a first case-driven screening study of soil-associated plastic debris \SI{<=2}{\milli\meter} in topsoil previously covered with plastic films. The screening revealed that the seasonal application of thin (\SI{40}{\micro\meter}) and perforated films may already lead to a 30-fold increase in soil plastic contents (up to \SI{35}{\micro\gram\per\gram}) compared to soil covered with thicker films (\SI{50}{\micro\meter}) and fleeces. \Ac{dsc} and \ac{ftir} analyses of the applied plastic covers further showed first signs of polymer aging that may have triggered the formation of plastic debris. However, the distribution of plastic debris in and around the covered agricultural fields did not allow for an unequivocal identification of plastic sources and the transport behavior of plastic debris. This is in line with recent findings by \citet{CorradiniMicroplastics2021} who screened different land use types for plastic debris at a regional level. While the authors found the majority of plastics on cropland and grassland, they were not able to track down the source of plastic pollution. In addition, most of the studies, including the one presented in Chapter~\ref{ch:screening}, have so far restricted their samplings to the uppermost \SI{5}{\centi\meter} of agricultural or industrial soils, quantified only selected polymers, and excluded nanoplastics.
An unequivocal source identification will most likely require a more systematic, in-depth monitoring. To facilitate this, the mass-based quantification of plastic debris should be further refined and extended in its application range, for instance, by making solvent-based \ac{py-gc-ms} amenable to other relevant polymers such as \ac{pet} and \ac{pvc} or biodegradable plastics and organic-rich environmental matrices like forest soil and compost. Sample preparation techniques for nanoplastics in soil are still to be developed, though. Complementary \ac{ftir} or Raman microspectroscopy could provide additional information on particle shapes and sizes, particularly for plastic hotspots (Chapter~\ref{ch:analytical-techniques}).
Yet, the insights gained in Chapter~\ref{ch:screening} suggest that the short-term use of thicker and more durable plastic covers should be preferred to limit future plastic emissions and accumulation in soil. Moreover, current EU regulations \citep{EN13655Plastics2018} and recycling efforts for agricultural plastics should be intensified to further reduce the amount of plastics unintentionally released into the environment.
Therefore, future research will not only need to systematically trace the major sources of plastic inputs back to different land use systems but also aim for a better understanding of the fluxes of plastic debris in and on soil as well as on landscape level.
% TODO: Cite \citep{BelzCurrent2021}?
% TODO: Add \citep{HermabessiereMicrowaveAssisted2021} that picked up my approach, including data analysis

\section{Fate of Plastic Debris in Soil and Beyond}
\label{sec:general-discussion:fate}

As outlined in Chapter~\ref{ch:intro}, soil and plastic debris share common characteristics like their particulate and polymeric structure. These structural similarities feed back to their fate.
Low-density \ac{som}, for instance, is well-known to preferentially erode from agricultural fields \citep{LalSoil2005,RumpelPreferential2006}. In the same way, fragments of low-density \ac{pe} or \ac{pp} covers ($\rho$ = \SIrange{0.9}{1.0}{\gram\per\cubic\centi\meter}) are transported by wind \citep{RezaeiWind2019,BullardPreferential2021} and water \citep{LaermannsTracing2021,RehmSoil2021} and are thus capable of traveling distances comparable to other non-volatile and water-soluble contaminants \citep{StubbinsPlastics2021}. \Citet{ZhangDistribution2020} estimated that \SI{96}{\percent} of \ac{ldpe} debris resting on the soil surface (\SI{5}{\degree} hillslope) readily runs off during heavy rainfall events (\SI{>300}{\milli\meter}) while the remaining \SI{4}{\percent} is retained in the soil. By contrast, polymers with higher densities ($\rho$ = \SIrange{1.3}{1.4}{\gram\per\cubic\centi\meter}) may be retained better due to gravitational sedimentation \citep{DongTransport2021,O'ConnorMicroplastics2019}.

Once deposited, the plastic debris is likely to age and fragment further (Chapter~\ref{ch:screening}), partially mix with soil, and become entrapped in soil pores or incorporated into soil aggregates \citep{RilligMicroplastic2017}. \Citet{ZhangDistribution2018} found \SI{72}{\percent} of plastic debris (\SIrange{0.05}{1}{\milli\meter}) associated with soil aggregates of an arable soil. The polymeric nature of both plastics and \ac{som} probably facilitates their mutual intermolecular stabilization and heteroaggregation \citep{SchaumannSoil2006,RenMicroplastics2021}. Due to an increased surface-to-volume ratio, small particles are most likely stabilized better than larger particles. In line with this, field-scale rainfall simulations showed that \SIrange{250}{300}{\micro\meter} \ac{hdpe} particles eroded faster than those of \SIrange{50}{100}{\micro\meter} size \citep{RehmSoil2021}.

Advective transport of plastic particles, including nanoplastics, through the soil column is often \SI{<50}{\percent} lower than conservative tracers and thus mostly limited to the uppermost \SIrange{10}{25}{\centi\meter} of the soil \citep{KellerTransport2020,WuTransport2020}. Instead, plastic debris is rather assumed to be transported to deeper soil via tillage, bioturbation, and preferential flow through soil cracks \citep{RilligMicroplastic2017a,YuLeaching2019,LiVertical2021}. The latter pathway might become increasingly relevant with more frequent and extreme weather events caused by climate change. Preferential flow may eventually make plastic particles reach groundwater where saturated conditions could promote their mobility\sidenote{See \citet{RenMicroplastics2021} for a review on plastic fate at the soil--groundwater interface.}. Under such saturated conditions, spherical particles were shown to be more mobile than fragmented or fibrous ones, particularly if the size of the plastic particles is lower than that of the medium \citep{WaldschlagerInfiltration2020}. Worst-case projections by \citet{WaldschlagerInfiltration2020} indicated a maximum infiltration depth of spheres in medium gravel of about \SI{2}{\meter} when applying an immense water flow of \SI{150}{\liter\per\minute\per\square\meter} for \SI{1}{\hour}. Plastic fragments and fibers infiltrated less than \SI{10}{\centi\meter} into matrices with smaller particles sizes like fine gravel, medium sand, or coarse silt.
This may also explain why elevated microplastic levels of up to \num{80}\,particles\,\si{\per\liter} groundwater have so far only been detected in the immediate vicinity of heavily contaminated sites such as landfills \citep{ManikandaBharathSpatial2021}. Groundwater wells were virtually free of plastics \citep{MintenigLow2019}.

Hence, current research suggests that:

\begin{itemize}
	\item Plastic particles larger than soil particles and less dense than the soil bulk density are preferentially eroded.
	\item Smaller plastic debris with a density close to the bulk density of the soil may rather be retained in the soil.
	\item Plastic retention in soil limits further transport and favors accumulation.
\end{itemize}

In fact, the continuous use of agricultural plastic covers as well as sewage sludge applications were already shown to accumulate plastic debris in agricultural topsoil \citep{HuangAgricultural2020,CorradiniEvidence2019}\sidenote{See \citet{BuksGlobal2020} for a comprehensive review on plastic debris in soil.}. In this sense, plastic debris may become an artificial \ac{som} fraction in the long term. \Citet{StubbinsPlastics2021}, for example, recently numbered plastic debris among soil \ac{Corg} stocks. This may particularly apply to aged plastics that will become more similar to the surrounding soil with time (Table~\ref{tab:polymer-aging}).
On the contrary, \citet{RilligMicroplastic2018} argued that plastic debris is too different in its physicochemical properties and function to be considered a part of \ac{som}. Due to the recalcitrance of conventional plastics towards degradation, they will if at all only participate in the soil's long-term carbon cycle, probably comparable to the pyrogenic biochar components of \latin{Terra preta} \citep{RilligMicroplastic2021}. The contemporary contribution of plastic debris to soil functions and quality may, however, be limited if not detrimental. Moreover, changing environmental conditions and agricultural practices like tillage have the potential of remobilizing deposited plastics. Aside from the plastics itself, their capability of releasing additives and other polymer-associated compounds into the environment potentially contributes to the adverse effects to soil quality.

\begin{margintable}[-15\baselineskip]
	\centering\footnotesize
	\caption[Changes in polymer properties while aging.]{Changes in polymer properties while aging \citep{RenMicroplastics2021}.}\label{tab:polymer-aging}
	\begin{tabular}{lc}
		\toprule
		Surface roughness & + \\
		Microcracks & + \\
		Tensile strength & \textminus \\
		Crystallinity & + \\
		Polarity & + \\
		Molecular weight & \textminus \\
		Functional groups & + \\
		(\ch{COOH}, \ch{C=O}, \ch{C-OH}, \ch{=CH}) \\
		Leaching of additives & + \\
		Sorption capacity  & + \\
		\bottomrule
	\end{tabular}
\end{margintable}

\section{Effects of Plastic Debris on Soil Quality}
\label{sec:general-discussion:effects}

While the effects of agricultural plastic covers on soil quality are extensively discussed in Chapter~\ref{ch:plastic-mulching}, the potential effects of plastic debris accumulating in soil were virtually unknown at the time of writing.
Current reviews point out that plastic debris have the ability of changing the soil's microstructural environment which may lead to adverse effects on the soil microbial community, plant growth, net primary production, and litter decomposition \citep{RilligMicroplastic2021,MbachuRise2021,QiBehavior2020}. \Citet{ZhangSystematical2021,RilligMicroplastic2021} reasoned that such effects may be either direct or indirect, namely mediated by changes in the soil's physicochemical properties. \Citet{deSouzaMachadoMicroplastics2019}, for instance, spiked a loamy sand at \SIlist{2;20}{\milli\gram\per\gram} \ac{pet} fibers, \ac{pa} beads, and \ac{pe}, \ac{pp}, \ac{ps}, and \ac{pet} fragments (\SIrange{8}{5000}{\micro\meter} size) and assessed various soil quality criteria after \SI{12}{weeks} of incubation. Fragments and fibers decreased the soil bulk density by up to \SI{20}{\percent} and thereby significantly altered the soil structure. The water holding capacity increased by \SIrange{10}{40}{\percent}; the effect was most pronounced for \ac{pet} fibers. Evapotranspiration and the fraction of water stable aggregates were reduced by up to \SIlist{60;25}{\percent}, respectively, in the presence of \ac{pet} fibers and \ac{pa} beads. These changes in the soil water dynamics are probably \ac{som}-dependent \citep{ZhangVariations2020,LiangEffects2021} and propagated to an increase in soil microbial activity\sidenote{See also \citet{deSouzaMachadoImpacts2018}.} and root and bulb growth of \latin{Allium fistulosum}, the spring onion \citep{deSouzaMachadoMicroplastics2019}. In general, the effects observed by \citet{deSouzaMachadoMicroplastics2019} were more pronounced for fibers and beads whose shape considerably differs from natural soil particles\sidenote{See also \citet{RilligMicroplastic2019,LehmannMicroplastics2021}.}.
Nonetheless, the applied plastic levels were about \numrange{2}{3} orders of magnitude higher than plastic contents measured in agricultural soil (\SI{<50}{\micro\gram\per\gram}, Chapter~\ref{ch:screening}), which challenges the environmental relevance of the reported, hardly detrimental effects. In line with this, a meta analysis by \citet{GaoEffects2019} only found adverse effects of residual plastic films on crop yield at levels \SI{>24}{\gram\per\square\meter} agricultural soil. However, inconsistent units impede further comparisons.

\Citet{BuksWhat2020} comprehensively reviewed the current state of research on the effects of plastic debris towards the multicellular soil fauna. The authors inferred that nematodes, gastropods, and rotifers responded the most sensitively to plastic levels \SI{<=100}{\micro\gram\per\gram} \citep[for instance,][]{KimSizedependent2020,SongUptake2019}. By contrast, collembolans and earthworms were more robust and required barely realistic plastic contents of \SI{>1}{\milli\gram\per\gram} to induce adverse effects \citep{JuEffects2019,DingEffect2021,LahiveMicroplastic2019}. Beetles, termites, ants, and mites remained largely unaffected by plastic debris \citep[for instance,][]{PengBiodegradation2019,ZhuTrophic2018}. The observed effects were mostly sublethal and included changes in the gut microbiome, oxidative stress, metabolic malfunctioning \citep{JuEffects2019,ChenDefense2020,Rodriguez-SeijoOxidative2018}, avoidance \citep{DingEffect2021}, or a reduced motility, growth, or reproduction \citep{BootsEffects2019,LahiveMicroplastic2019}. In line with the findings by \citet{deSouzaMachadoMicroplastics2019} discussed above, smaller and more irregularly shaped particles like fragments and fibers tended to be more ecotoxicologically active than larger and spherical particles; yet, \ac{ps} microbeads still dominate in effect studies \citep{BuksWhat2020}. The polymer type appears to be less influential \citep{RilligMicroplastic2020}, which may be different for nanosized particles, though \citep{RilligMicroplastic2019}.

\citet{BahoMicroplastics2021} criticized that the vast majority of effect studies have focused on single species so far and were rather short-term (\SI{30}{\day}). The authors advocated for longer-term multispecies and ecosystem level studies to prevent an ``ecological surprise''; this is when ecosystems behave fundamentally different than previously anticipated from smaller-scale laboratory or mesocosm studies. Despite these urges for more realistic testing, the majority of studies applied plastic contents far beyond realistic exposure levels \citep{BuksWhat2020}. The actual contribution of plastic debris to the degradation of soil ecosystem and functioning therefore remains incompletely understood. This uncertainty is also because effect studies typically refer to plastic contents on a mass basis while the greater part of exposure studies reports particle counts \citep{LeuschConverting2021}.

A first risk assessment recently conducted by \citet{JacquesProbabilistic2021} estimated environmental exposure distributions ($n = 48$) and species sensitivity distributions ($n = 37$) by converting published no and lowest observed effect particle masses to counts. Particle sizes or shapes were not considered. The hazardous plastic levels at which \SI{5}{\percent} of species would be affected were \numlist{230;160}\,particles\,\si{\per\kilo\gram} based on no and lowest observed effect levels, respectively. These species were \latin{Caenorhabditis elegans}, a model nematode, garden cress (\latin{Lepidium sativum}), and \latin{Aspergillus flavus}, a pathogenic fungus. The hazardous levels were exceeded in about \SI{60}{\percent} of the investigated exposure studies. At \num{8100}\,particles\,\si{\per\kilo\gram}, this is the \num{95}th percentile of the environmental exposure distribution, \SIrange{22}{28}{\percent} of species would be affected. The three sites exceeding that level were industrial, urban, and agricultural.
While this is certainly alarming, the results contrast previous mass-based studies that scarcely identified adverse effects on soil organisms at realistic exposure levels \citep{BuksWhat2020}. The distributions calculated by \citet{JacquesProbabilistic2021} highly depend on the quality of the input data and are subject to great uncertainties due to the underlying mass-to-particle conversions\sidenote{See also Chapters~\ref{ch:analytical-techniques} and \ref{ch:screening}.}. For example, some of the included studies conducted their ecotoxicity tests in aqueous soil solution, did not state the polymer used, or reported plastic levels per unit area.
Therefore, more systematic and unit-harmonized research is indispensable for a comprehensive eco(toxico)logical risk assessment. This should then also acknowledge plastic shapes, sizes, and aging and include interactive effects with other stressors like polymer-associated compounds.

\section[Interaction of Plastic Debris with Plastic-Associated Compounds]{Interaction of Plastic Debris with Additives, Agrochemicals, and other Plastic-Associated Compounds}
\label{sec:general-discussion:pacs}

The interaction of plastic debris with organic substances, be it additives or (agro)\-chemicals from external sources, is still understudied and to some extent contradictory\sidenote{See \citet{XiangMicroplastics2022} for a detailed overview.}.
Although the majority of polymers including agricultural plastic covers (Chapters~\ref{ch:plastic-mulching} and \ref{ch:screening}) contains additives like color pigments, antioxidants and light stabilizers, plasticizers, slip agents, or flame retardants \citep{HahladakisOverview2018}, their leaching from the polymer backbone and release into the surrounding soil is often hypothesized \citep[Chapter~\ref{ch:plastic-mulching};][]{PathanSoil2020,ZhangTransport2020} yet hardly observed in field studies \citep{QiBehavior2020}. \Citet{LiAre2021}, for instance, found up to \num{5} times higher \ac{pae} contents in greenhouse soil than in non-covered soil. But no clear correlation was established with the amount or type of plastic debris found, suggesting another input source or factor driving their distribution\sidenote{See also \citet{BillingsPlasticisers2021} for a comprehensive review on plasticizers in the terrestrial environment.}. This could be seed coatings, impure agrochemicals, or changing environmental conditions remobilizing legacy contaminations.
Other additives than \acp{pae} are typically added to polymers in much smaller quantities \citep{HahladakisOverview2018} so that their detection after desorbing and dispersing into the surrounding is particularly challenging.
% TODO: Add \citep{ZhangAgricultural2021}

In contrast to polymer additives desorbing from the polymer backbone, sorption of organic compounds to plastic particles has been more systematically investigated. \Citet{HufferSorption2016}, for example, studied the sorption behavior of numerous apolar, monopolar, and bipolar organic pollutants to \ac{pe}, \ac{pvc}, \ac{pa}, and \ac{ps} particles in aqueous solution. Aromatic compounds were shown to preferentially interact with \ac{ps} via $\pi$--$\pi$ stacking. Interactions with \ac{pe} were mostly driven by migration of the organic compounds into the polymer bulk phase, while surface interactions dominated with \ac{pa} and \ac{pvc}. Similarly, \iac{pe} mulch sorbed about \SI{23}{\percent} of various agrochemicals in aqueous solution which additionally slowed down their degradation by \SI{30}{\percent} \citep{BeriotLaboratory2020}. The sorption potential correlated well with the \ac{kow} of the agrochemicals \citep{BeriotLaboratory2020,WangAdsorption2020,SuntaAdsorption2020,LanComparative2021}. The molecular mechanisms controlling the sorption process were hydrophobic partitioning and electrostatic forces \citep{TourinhoPartitioning2019,LanComparative2021}.
However, \citet{RamosPolyethylene2015} already showed that migration of pesticides to\slash from \ac{pe} covers may be bidirectional. The state of the dynamic equilibrium may depend on the organic substance of interest and its polarity (\ac{kow}) as well as the polymer type, the degree of polymer aging, interactions with other environmental compartments, and prevailing environmental conditions. An increase in the soil water content, for instance, may mobilize the disinfectant triclosan previously sorbed to \ac{pe} \citep{ChenComparison2021}.

In aqueous solution, \ac{uv}-aged \ac{ps} particles sorbed up to \num{10} times less organic pollutants than virgin particles \citep{HufferData2018,HufferSorption2018}. This was attributed to surface oxidation of the polymer surface as indicated by an increased number of carbonyl moieties. By contrast, the broad-spectrum insecticide fibronil sorbed best to polymers containing carbonyl groups \citep{GongComparative2019}.

An agricultural sandy loam spiked at \SI{100}{\milli\gram\per\gram} \ac{pe} reduced the soil's overall sorption capacity towards the herbicides atrazine and 2,4-D by half compared to blank soil \citep{HufferPolyethylene2019}. However, this effect was most pronounced at pH~\numrange{3}{5}, and \ac{pe} contents were particularly high. In line with this, \SI{100}{\milli\gram\per\gram} \ac{pe}, \ac{pp}, and \ac{ps} decreased the sorption of diazepam, an anxiolytic, to sandy loam and silty clay. At \SI{10}{\milli\gram\per\gram} \ac{pe}, \ac{pp}, and \ac{ps}, however, the sorption potential of phenanthrene increased. Phenanthrene sorbed best to \ac{pe}, followed by \ac{som}, \ac{pp}, and \ac{ps}. Polymers pretreated with \ac{som} were decreased in their sorption capacity \citep{XuContrasting2021}.
By contrast, \citet{AteiaSorption2020} assessed the interaction of atrazine, acetamidophenol, and two perfluoroalkyl compounds with eight different polymers and kaolin in the presence of dissolved organic matter. The sorption to kaolin was generally about one order of magnitude lower than the sorption to polymers. Recycled plastics and polymers inoculated for \num{2}~weeks with organic matter sorbed a higher amount of organic substances than virgin polymers.
In line with this, sorption to \ac{pe} debris reduced the adverse effects of phenanthrene and anthracene towards bacteria \citep{KleinteichMicroplastics2018}. On the contrary, the bioconcentration of perfluoroalkyl compounds in earthworms doubled in the presence of \ac{pvc} particles \citep{SobhaniMicroplastics2021}.

Current literature indicates that the desorption and release of polymer additives into the surrounding soil requires more basic research. The sorption of other, non-additive organic compounds like agrochemicals or pharmaceuticals to plastic particles is understood better but their competitive sorption potential in the presence of different soil constituents and polymer types should be assessed more systematically. All in all, the sorption behavior of organic substances to plastic particles depends on the polarity of the organic substance (\ac{kow}), the quantity and type of the polymer present, and the soil characteristics including its texture, water content and cation exchange capacity.
In comparison to the plastic contents reported in Chapter~\ref{ch:screening} (\SI{<50}{\micro\gram\per\gram}), sorption studies have so far used \num{>20} times higher plastic contents. As opposed to the assumptions made in Chapter~\ref{ch:plastic-mulching}, such low plastic contents are suggested to have only a limited influence on the soil's overall (de)sorption potential. Nanoplastics with particularly high surface areas may need to be separately assessed, though. Moreover, it is still unknown how plastic aging may affect the fate of polymer-associated compounds in the long term.
% TODO: Add \citep{CastanMicroplastics2021}

\section{Towards a Sustainable Use of Agricultural Plastic Covers?}
\label{sec:general-discussion:sustainable}

Given the current uncertainties about the long-term fate and effects of plastic debris in the terrestrial ecosystem, a sustainable agriculture may be advised to act according to the precautionary principle \citep{RhodesPlastic2018,BackhausMicroplastics2020,MollerFinding2020}. This is not only to protect the environment but also agronomic revenues. \Citet{BertlingKunststoffe2021} recently stated that German agricultural fields with plastic levels \SI{>1}{\milli\gram\gram} become unattractive to farmers and thus practically loose their agronomic value. This coincides with the effect levels discussed in Section~\ref{sec:general-discussion:effects}, however, subtle or long-term effects on the soil ecosystem may already occur at plastic levels \SI{<1}{\milli\gram\per\gram}. Overstepping certain, yet unknown ``microplastic tipping points'' is to be avoided \citep{QiBehavior2020}.

In order to limit future plastic emissions into the environment, \citet{ThompsonPlastics2009,ScalengheResource2018,RhodesPlastic2018} proposed a holistic systems thinking approach that integrates plastic production, consumption, and disposal into a sustainable circular economy. This means:

\begin{enumerate}
	\item Reducing plastic products and, if possible, replacing them with viable alternatives.
	\item (Re)using plastic materials as long as possible.
	\item Recycling the plastics at their end of life.
\end{enumerate}

\Citet{SimonBinding2021} additionally advocated for a greater durability of newly produced plastic products that would virtually eliminate single use, an increased product safety, as well as a transparent labeling of biodegradable and bio-based plastics. While biodegradable polymers are specifically designed to mineralize after a certain period of time under predefined conditions, bio-based plastics are conventional polymers stemming from renewable instead of fossil resources.

Simply reducing the application range of agricultural covers made from conventional plastics would probably decrease the global agricultural production (Chapter~\ref{ch:plastic-mulching}). Therefore, replacing conventional plastics with biodegradable alternatives or other, natural materials seems more practicable \citep{BrandesMikro2020} and is, in fact, increasingly implemented. Agricultural covers made of biodegradable polymers like \ac{pla} or starch blends already capture a market share of \SI{7}{\percent} \citep{BertlingKunststoffe2021}. However, the main challenge in the development of biodegradable agricultural covers remains: This is making them resistant to degradation while in use but ensuring their complete degradation under realistic environmental conditions at the end of life \citep[Chapter~\ref{ch:plastic-mulching};][]{BertlingKunststoffe2021}. The use of biodegradable plastic covers thus continues to bear the risk of leaving incompletely degraded plastic debris in the soil after the growing season \citep{SanderBiodegradation2019,VieraAre2021}.

At the same time, conventional \ac{pe} mulches covering asparagus or strawberries cultivations (\SI{>50}{\micro\meter} film thickness) are already reused for up to \num{10}~years. Thinner perforated foils or fleeces have shorter life times and are often replaced after one growing season \citep[Chapter~\ref{ch:screening};][]{BertlingKunststoffe2021}. Therefore, the use of thicker covers may not only reduce the formation of plastic debris but also their potential for reuse.

The recycling of used plastic covers was challenging in the past for their contamination with soil and agrochemicals (Chapter~\ref{ch:plastic-mulching}). However, advances by the German recycling system ``ERDE'' recently enabled the retrieval, collection, and mechanical recycling of \SI{50}{\percent} of all agricultural plastic covers used across the country \citep{BertlingKunststoffe2021,ERDERecyclingERDE2021}.

Life cycle assessments show that conventional, petrochemically produced \ac{pe} mulches emit \SIlist{2.7;3.6}{\kilo\gram} \ch{CO2} \si{\per\kilo\gram} mulch when landfilled or incinerated, respectively. Mechanical recycling has a carbon footprint of \SI{1.2}{\kilo\gram} \ch{CO2} \si{\per\kilo\gram} \ac{pe} mulch \citep{BosLife2008}. In contrast, biodegradable and bio-based alternatives like \ac{pla} have the potential to close material cycles and thus reduce their carbon footprint to up to \SI{1.7}{\kilo\gram} \ch{CO2} \si{\per\kilo\gram} \citep{KollerSwitching2019,RezvaniGhomiLife2021}. At least for \ac{pla}, however, polymer production is still energy intensive and may cause \ch{CO2} emissions of up to \SI{2.8}{\kilo\gram\per\kilo\gram} \ac{pla} which is only compensated if the \ac{pla} is not left for degradation but mechanically recycled \citep{RezvaniGhomiLife2021,AltmanMyth2021}. Further research and development is thus needed to make the production of biodegradable and bio-based polymers more energy efficient \citep{VieraAre2021}. This may also include the intelligent design of novel polymers that are optimized towards low energy production and recycling \citep{VieraAre2021,KorleyPolymer2021,KakadellisAchieving2021} while avoiding competition with food supply \citep{RhodesPlastic2018}. Moreover, it is still largely unknown how and to what extent biodegradable agricultural covers and plastic debris at various stages of decomposition may affect the soil ecosystem \citep{SanderBiodegradation2019,QinReview2021,AltmanMyth2021}. Adding fresh and easily available carbon sources to soil is well known to stimulate the decomposition of older and more recalcitrant \ac{som}; a phenomenon called priming \citep{ChenMixing2020}.

For agricultural plastics escaping the material cycle and entering the environment, \citet{BertlingKunststoffe2021} suggests the implementation of some sort of ``emergency degradation'' that ensures the decomposition of plastics lost in the field after at least 50~years. This, as well, would imply the development of new polymers or the modification of biodegradable polymers to resist degradation for a longer period of time. On the contrary, \citet{ScalengheResource2018} raised the question whether it may be generally acceptable to leave certain amounts of chemically-inert plastics in the environment. This idea may have been motivated by the fact that the remediation and removal of plastic debris from soil is still at an early stage of development \citep{PadervandRemoval2020} and will most likely come with potential unwanted side effects.

For the lack of ready-to-use alternatives, a sustainable use of agricultural plastic covers should first and foremost aim to reduce plastic emissions by using less or more durable plastic covers that prevent the emission of plastic debris and thus long-term plastic accumulation in the environment.
The transition to a more plastic-aware agriculture could be guided by targeted policy and regulatory measures. While the EU already banned agricultural plastic covers thinner than \SI{20}{\micro\meter} (Chapter~\ref{ch:screening}), \citet{StubenrauchPlastic2020} recently elaborated that the current ``command-and-control legislation'' by the EU and Germany insufficiently protect agricultural soil from plastic emissions. Their analysis comes to the conclusion that the current legislation particularly neglects the monitoring and regulation of plastic debris \SI{<2}{\milli\meter} which has so far been justified with restricted analytical capabilities. The analytical advances outlined in this dissertation thesis may thus eventually contribute to a better regulation of plastic debris in soil.
% TODO: Add \citep{DemeMacro2022}
% TODO: Problem PLA, eigentlich nicht richtig abbaubar auf dem Feld, eher bei 60--70 °C im Komposter. Hier könnte eine ``Notabbaubarkeit'' nach wenigen Jahrzehnten eher gewährleistet sein.
% Add \citep{WangBiodegradable2021}

\section{Concluding Thoughts}
\label{sec:general-discussion:conclusion}

The aim of this dissertation thesis was to scrutinize the extent to which agricultural plastic covers emit plastic debris into the surrounding soil. To this end, several thermoanalytical techniques were assessed, developed, and validated. The final solvent-based \ac{py-gc-ms} method enabled the simultaneous and selective routine analysis of \ac{pe}, \ac{pp}, and \ac{ps} debris in soil. It further had the advantage of delivering highly sensitive mass-based information that can easily be compared with modeling or effect data. Applying the method in a first screening study indeed indicated that plastic covers function as a source of plastic debris in agricultural topsoil. However, the found plastic levels were considerably lower than in previous reports and several orders of magnitude lower than effect levels. At the current state of research, the risk of plastic debris to soil organisms is thus limited but may increase with more and nanosized plastics accumulating in the environment. This calls for precaution particularly because current literature suggests that adverse effects of plastic debris on soil life and quality largely depend on the size and shape of the plastics. Particle shapes and sizes cannot be assessed with the mass-based approach taken in this thesis and require complementary microspectroscopy of potential hotspots.
Furthermore, the distribution processes and material fluxes of plastic debris in the terrestrial environment are still incompletely understood and need to be assessed in future, more systematic research. This should extend to other polymers including biodegradable plastics and nanoplastics which still lack appropriate analytical techniques.

% TODO: Abschnitt ``Future of analytical techniques'' oder Verweis auf Kapitel 5?
